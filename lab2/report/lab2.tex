\documentclass[12pt]{article}

\usepackage[utf8]{inputenc}
\usepackage[T1, T2A]{fontenc}
\usepackage{fullpage}
\usepackage{multicol, multirow}
\usepackage{tabularx}
\usepackage{ulem}
\usepackage{listings}
\usepackage[english, russian]{babel}
\usepackage{tikz}
\usepackage{pgfplots}
\usepackage{indentfirst}
\usepackage{noindentafter}
\usepackage{nonfloat}
\usepackage{ulem}
\usepackage{fancyhdr}
% \usepackage{courier}
% \usepackage{FiraMono}
\usepackage{color}
\usepackage{subcaption}
\usepackage{titlesec}


\parindent=1cm

\linespread{1}
\pgfplotsset{compat=1.16}
\newcommand{\se}[1]{\section*{\bf #1}}

\newcommand{\listsource}[2]{
	\subsection*{\textbf{#2}}
	\lstinputlisting[language=c++]{#1/#2}
}
\newcommand{\listtext}[2]{
	\subsection*{\textbf{#2}}
	\lstinputlisting[basicstyle=\tiny]{#1/#2}
}

\lstdefinestyle{empty}{language=c++,
	basicstyle=\scriptsize,
	showspaces=false,
	showstringspaces=false,
	showtabs=false, вариантов
	tabsize=4,
	breaklines=true,
	escapechar=@,
	numbers=none,
	frame=none,
	escapeinside={\%*}{*)},
	breakatwhitespace=false % переносить строки только если есть пробел
}

\lstdefinestyle{customc}{
	belowcaptionskip=1\baselineskip,
	breaklines=true,
	frame=L,
	xleftmargin=\parindent,
	numbers=left,
	showstringspaces=false,
	basicstyle=\footnotesize\ttfamily,
	keywordstyle=\bfseries\color{green!40!black},
	commentstyle=\itshape\color{purple!40!black},
	identifierstyle=\color{blue},
	stringstyle=\color{orange},
}

\lstset{escapechar=|,style=customc}

\pgfplotsset{compat=1.17}

\titleformat{\section}
{\normalfont\Large\bfseries}{\thesection.}{0.3em}{}

\titleformat{\subsection}
{\normalfont\large\bfseries}{\thesubsection.}{0.3em}{}

% \titlespacing{\section}{0pt}{*2}{*2}
% \titlespacing{\subsection}{0pt}{*1}{*1}
% \titlespacing{\subsubsection}{0pt}{*0}{*0}
% \lstloadlanguages{Lisp}
% \lstset{extendedchars=false,
% 	escapechar= |,
% 	breaklines=true,
% 	breakatwhitespace=true,
% 	keepspaces = true,
% 	tabsize=2
% }

\newcommand{\makemytitlepage}[2]{
	\thispagestyle{empty}
	\begin{center}
		\bfseries

		{\Large Московский авиационный институт\\ (национальный исследовательский университет)

		}

		\vspace{48pt}

		{\large
			Институт №8 ``Информационные технологии и прикладная математика''
			Кафедра 806 ``Вычислительная математика и программирование''
		}

		\vspace{36pt}

		{Лабораторная работа №#1  \\
			По курсу ``Программирование графических процессоров''
		}
		\vspace{48pt}

		{#2}

	\end{center}

	\vspace{72pt}

	\begin{flushright}
		\begin{tabular}{rl}
			Выполнил:      & П.\, А. Милько      \\
			Группа:        & М8О-408Б-17         \\
			Преподаватели: & К.Г. Крашенинников, \\
			               & А.Ю. Морозов.       \\
		\end{tabular}
	\end{flushright}

	\vfill

	\begin{center}
		\bfseries
		Москва\\
		\the\year
	\end{center}
	\newpage
	\setcounter{page}{1}
}

\newcommand{\nvidia}[0]{
	\se{Программное и аппаратное обеспечение}

	Редактор: NVIM v0.4.4

	GPU:

	\begin{tabular}{ll}
		Driver Version  & : 460.39          \\
		CUDA Version    & : 11.2            \\
		Product Name    & : GeForce GT 740M \\
		VBIOS Version   & : 80.28.22.00.23  \\
		Total Memory    & : 2004 MiB        \\
		Graphics Clocks & : 980 MHz         \\
		Memory Clocks   & : 920 MHz         \\
	\end{tabular}


}

\begin{document}
\makemytitlepage{2}{Обработка изображений на GPU. Фильтры.}

\se{Цель работы}

Научиться использовать GPU для обработки изображений.
Использование текстурной памяти.

\textbf{Вариант 6.} Выделение контуров. Метод Превитта.

В качестве вещественного типа данных необходимо использовать тип данных
double. Все результаты выводить с относительной точностью $10^{-10}$ . Ограничение:
$n < 2^{25}$

\nvidia

\se{Метод решения}
Создать результирующий вектор по размеру входного, пробежаться по исходному вектору
и записать значения в результирующий с индексом $n-1$.

\se{Описание программы}
Единственный файл \lstinline|main.cu|.

Реализованное ядро называется \lstinline|inverse|. Содержит единственный цикл с тривиальным условием.

% \listsource{..}{main.cu}
% \listsource{..}{main.c}

\newpage

\se{Результаты}

\begin{table*}[!htb]
	\begin{subtable}{.333\linewidth}
		\caption{$10^3$ элементов}
		\centering

		\lstinline|cpu time = 0.007000|

		\begin{tabular}{|c|c|c|}
			\hline
			blocks & threads & time       \\
			\hline

			128    & 128     & 000.036640 \\
			128    & 32      & 000.037088 \\
			32     & 128     & 000.039008 \\
			32     & 32      & 000.041248 \\
			32     & 1024    & 000.041920 \\
			1024   & 128     & 000.043584 \\
			1      & 32      & 000.046112 \\
			128    & 1024    & 000.047040 \\
			128    & 1       & 000.047520 \\
			1024   & 1       & 000.050208 \\
			32     & 1       & 000.054240 \\
			1024   & 32      & 000.057536 \\
			1      & 1024    & 000.066880 \\
			1      & 128     & 000.076160 \\
			1      & 1       & 000.149152 \\
			1024   & 1024    & 000.163264 \\
			\hline
		\end{tabular}
	\end{subtable}
	\begin{subtable}{.333\linewidth}
		\caption{$10^5$ элементов}
		\centering

		\lstinline|cpu time = 0.200000|

		\begin{tabular}{|c|c|c|}
			\hline
			blocks & threads & time       \\
			\hline

			128    & 128     & 000.037600 \\
			32     & 1024    & 000.040832 \\
			32     & 128     & 000.042400 \\
			1024   & 128     & 000.046528 \\
			128    & 1024    & 000.055168 \\
			1      & 1024    & 000.067008 \\
			128    & 32      & 000.091232 \\
			1024   & 32      & 000.093760 \\
			32     & 32      & 000.117824 \\
			1024   & 1024    & 000.170272 \\
			1      & 128     & 000.444320 \\
			1      & 32      & 001.752160 \\
			1024   & 1       & 001.756608 \\
			128    & 1       & 001.949856 \\
			32     & 1       & 003.432608 \\
			1      & 1       & 011.677664 \\
			\hline
		\end{tabular}
	\end{subtable}
	\begin{subtable}{.333\linewidth}
		\caption{$10^6$ элементов}
		\centering

		\lstinline|cpu time = 1.997000|

		\begin{tabular}{|c|c|c|}
			\hline
			blocks & threads & time       \\
			\hline

			1024   & 128     & 000.273600 \\
			32     & 1024    & 000.277280 \\
			128    & 1024    & 000.290944 \\
			128    & 128     & 000.293120 \\
			32     & 128     & 000.345056 \\
			1024   & 1024    & 000.441344 \\
			1      & 1024    & 000.578688 \\
			1024   & 32      & 000.763008 \\
			128    & 32      & 000.833280 \\
			32     & 32      & 001.102944 \\
			1      & 128     & 004.359520 \\
			1      & 32      & 017.438175 \\
			1024   & 1       & 017.528929 \\
			128    & 1       & 025.771551 \\
			32     & 1       & 034.263199 \\
			1      & 1       & 116.714722 \\
			\hline
		\end{tabular}
	\end{subtable}

\end{table*}

\se{Выводы}

Всю жизнь я делал реверс вектора в один поток и мне даже не приходила в голову идея
этот алгоритм распараллелить. Как оказалось, это имееет смысл для GPU.

Иногда нужно производить быстрый реверс вектора, например для подготовки данных для
обучения нейросети.

Судя по результатам тестов не всегда имееет смысл отдавать все доступные ресурсы на
выполнение операции. В тесте с небольшим количеством элементов ($10^3$)
вариант с \lstinline|<<<1024,1024>>>| показал худший результат из всех. Больше времени ушло на создание потоков,
чем на выполнение кода.

В итоге, GPU оказалась быстрее чем CPU примерно в 7 раз (на самом большом тесте).
Из этого можно сделать вывод, что использование GPU не целесообразно для мизерного набора данных.

\end{document}