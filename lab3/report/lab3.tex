\documentclass[12pt]{article}

\usepackage{amsmath}

\usepackage[utf8]{inputenc}
\usepackage[T1, T2A]{fontenc}
\usepackage{fullpage}
\usepackage{multicol, multirow}
\usepackage{tabularx}
\usepackage{ulem}
\usepackage{listings}
\usepackage[english, russian]{babel}
\usepackage{tikz}
\usepackage{pgfplots}
\usepackage{indentfirst}
\usepackage{noindentafter}
\usepackage{nonfloat}
\usepackage{ulem}
\usepackage{fancyhdr}
% \usepackage{courier}
% \usepackage{FiraMono}
\usepackage{color}
\usepackage{subcaption}
\usepackage{titlesec}


\parindent=1cm

\linespread{1}
\pgfplotsset{compat=1.16}
\newcommand{\se}[1]{\section*{\bf #1}}

\newcommand{\listsource}[2]{
	\subsection*{\textbf{#2}}
	\lstinputlisting[language=c++]{#1/#2}
}
\newcommand{\listtext}[2]{
	\subsection*{\textbf{#2}}
	\lstinputlisting[basicstyle=\tiny]{#1/#2}
}

\lstdefinestyle{empty}{language=c++,
	basicstyle=\scriptsize,
	showspaces=false,
	showstringspaces=false,
	showtabs=false, вариантов
	tabsize=4,
	breaklines=true,
	escapechar=@,
	numbers=none,
	frame=none,
	escapeinside={\%*}{*)},
	breakatwhitespace=false % переносить строки только если есть пробел
}

\lstdefinestyle{customc}{
	belowcaptionskip=1\baselineskip,
	breaklines=true,
	frame=L,
	xleftmargin=\parindent,
	numbers=left,
	showstringspaces=false,
	basicstyle=\footnotesize\ttfamily,
	keywordstyle=\bfseries\color{green!40!black},
	commentstyle=\itshape\color{purple!40!black},
	identifierstyle=\color{blue},
	stringstyle=\color{orange},
}

\lstset{escapechar=|,style=customc}

\pgfplotsset{compat=1.17}

\titleformat{\section}
{\normalfont\Large\bfseries}{\thesection.}{0.3em}{}

\titleformat{\subsection}
{\normalfont\large\bfseries}{\thesubsection.}{0.3em}{}

% \titlespacing{\section}{0pt}{*2}{*2}
% \titlespacing{\subsection}{0pt}{*1}{*1}
% \titlespacing{\subsubsection}{0pt}{*0}{*0}
% \lstloadlanguages{Lisp}
% \lstset{extendedchars=false,
% 	escapechar= |,
% 	breaklines=true,
% 	breakatwhitespace=true,
% 	keepspaces = true,
% 	tabsize=2
% }

\newcommand{\makemytitlepage}[2]{
	\thispagestyle{empty}
	\begin{center}
		\bfseries

		{\Large Московский авиационный институт\\ (национальный исследовательский университет)

		}

		\vspace{48pt}

		{\large
			Институт №8 ``Информационные технологии и прикладная математика''
			Кафедра 806 ``Вычислительная математика и программирование''
		}

		\vspace{36pt}

		{Лабораторная работа №#1  \\
			По курсу ``Программирование графических процессоров''
		}
		\vspace{48pt}

		{#2}

	\end{center}

	\vspace{72pt}

	\begin{flushright}
		\begin{tabular}{rl}
			Выполнил:      & П.\, А. Милько      \\
			Группа:        & М8О-408Б-17         \\
			Преподаватели: & К.Г. Крашенинников, \\
			               & А.Ю. Морозов.       \\
		\end{tabular}
	\end{flushright}

	\vfill

	\begin{center}
		\bfseries
		Москва\\
		\the\year
	\end{center}
	\newpage
	\setcounter{page}{1}
}

\newcommand{\nvidia}[0]{
	\se{Программное и аппаратное обеспечение}

	Редактор: NVIM v0.4.4

	GPU:

	\begin{tabular}{ll}
		Driver Version  & : 460.39          \\
		CUDA Version    & : 11.2            \\
		Product Name    & : GeForce GT 740M \\
		VBIOS Version   & : 80.28.22.00.23  \\
		Total Memory    & : 2004 MiB        \\
		Graphics Clocks & : 980 MHz         \\
		Memory Clocks   & : 920 MHz         \\
	\end{tabular}


}

\begin{document}
\makemytitlepage{3}{Классификация и кластеризация изображений на GPU.}

\se{Цель работы}

Научиться использовать GPU для классификации и
кластеризации изображений. Использование константной памяти.

\textbf{Вариант 5.} Метод k-средних.

\textbf{Входные данные.}
На первой строке задается путь к исходному изображению,
на второй, путь к конечному изображению. На следующей строке, число $nc$ -- кол-во
кластеров. Далее идут $nc$ строчек описывающих начальные центры кластеров. Каждая
i-ая строчка содержит пару чисел -- координаты пикселя который является центром.
$nc <= 32$.

\nvidia

\se{Метод решения}


\se{Описание программы}



{
	\scriptsize
	\listsource{../src}{main.cu}
}

\newpage

\se{Результаты}


\begin{figure}[tbh]
	\caption*{Малое изображение: 400x408}
	\begin{subfigure}{0.49\textwidth}
		\centering
		\caption*{Исходное изображение}
		\includegraphics[scale=0.4]{../test/images/nigger.png}
	\end{subfigure}
	\begin{subfigure}{0.49\textwidth}
		\centering
		\caption*{Результат:}
		\includegraphics[scale=0.4]{../test/t3/res.png}
	\end{subfigure}
\end{figure}

Были выделены пиксели с отдельных лепестков и из центральной области.

\begin{figure}[tbh]
	\caption*{Большое изображение: 6000x4000}
	\begin{subfigure}{0.49\textwidth}
		\centering
		\caption*{Исходное изображение}
		\includegraphics[scale=0.159]{../test/images/pumpkin.png}
	\end{subfigure}
	\begin{subfigure}{0.49\textwidth}
		\centering
		\caption*{Результат:}
		\includegraphics[scale=0.038]{../test/t4/res.png}
	\end{subfigure}
\end{figure}

Были выделены пиксели на облаке, чистом небе, тыкве и соломе.
Но так получилось что цвет соломы оказался несколько ближе к цвету тыквы.

\textit{Время указано в миллисекундах}


% \begin{center}
% 	cpu time = 348.877649
% 	\begin{table*}[!htb]
% 		\centering
% 		\begin{tabular}{|c|c|c|}
% 			\hline
% 			blocks    & threads & time        \\
% 			\hline
% 			16x16     & 16x16   & 005.450240  \\
% 			32x32     & 16x16   & 005.528064  \\
% 			8x8       & 8x8     & 005.688256  \\
% 			16x16     & 32x32   & 005.862272  \\
% 			64x64     & 16x16   & 005.932992  \\
% 			32x32     & 32x32   & 006.667456  \\
% 			4x4       & 4x4     & 021.764448  \\
% 			1024x1024 & 16x16   & 031.003839  \\
% 			1024x1024 & 32x32   & 204.964478  \\
% 			1x1       & 1x1     & 3699.413574 \\
% 			\hline
% 		\end{tabular}
% 	\end{table*}
% \end{center}


\se{Выводы}

Я понял что обрабатывать изображения на CPU плохая идея, если даже на одном небольшом изображении
разница получилась в 64 раза.

Конкретно выделение контуров было мне нужно для подготовки данных для обучения нейросети.
Нужно было разделить изображение на несколько по указанным линиям.
Правда там я использовал повышение контраста и резкости,
но чувствую было бы проще если бы я использовал предназначенный для этого алгоритм.

\end{document}