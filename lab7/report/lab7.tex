\documentclass[12pt]{article}

\usepackage{amsmath}


\usepackage[utf8]{inputenc}
\usepackage[T1, T2A]{fontenc}
\usepackage{fullpage}
\usepackage{multicol, multirow}
\usepackage{tabularx}
\usepackage{ulem}
\usepackage{listings}
\usepackage[english, russian]{babel}
\usepackage{tikz}
\usepackage{pgfplots}
\usepackage{indentfirst}
\usepackage{noindentafter}
\usepackage{nonfloat}
\usepackage{ulem}
\usepackage{fancyhdr}
% \usepackage{courier}
% \usepackage{FiraMono}
\usepackage{color}
\usepackage{subcaption}
\usepackage{titlesec}


\parindent=1cm

\linespread{1}
\pgfplotsset{compat=1.16}
\newcommand{\se}[1]{\section*{\bf #1}}

\newcommand{\listsource}[2]{
	\subsection*{\textbf{#2}}
	{\footnotesize
		\lstinputlisting[language=c++]{#1/#2}
	}
}

\lstdefinestyle{empty}{language=c++,
	basicstyle=\scriptsize,
	showspaces=false,
	showstringspaces=false,
	showtabs=false, вариантов
	tabsize=4,
	breaklines=true,
	escapechar=@,
	numbers=none,
	frame=none,
	escapeinside={\%*}{*)},
	breakatwhitespace=false % переносить строки только если есть пробел
}

\lstdefinestyle{customc}{
	belowcaptionskip=1\baselineskip,
	breaklines=true,
	frame=L,
	xleftmargin=\parindent,
	language=c++,
	numbers=left,
	showstringspaces=false,
	basicstyle=\footnotesize\ttfamily,
	keywordstyle=\bfseries\color{green!40!black},
	commentstyle=\itshape\color{purple!40!black},
	identifierstyle=\color{blue},
	stringstyle=\color{orange},
}

\lstset{escapechar=@,style=customc}

\pgfplotsset{compat=1.17}

\titleformat{\section}
{\normalfont\Large\bfseries}{\thesection.}{0.3em}{}

\titleformat{\subsection}
{\normalfont\large\bfseries}{\thesubsection.}{0.3em}{}

% \titlespacing{\section}{0pt}{*2}{*2}
% \titlespacing{\subsection}{0pt}{*1}{*1}
% \titlespacing{\subsubsection}{0pt}{*0}{*0}
% \lstloadlanguages{Lisp}
% \lstset{extendedchars=false,
% 	escapechar= |,
% 	breaklines=true,
% 	breakatwhitespace=true,
% 	keepspaces = true,
% 	tabsize=2
% }

\newcommand{\makemytitlepage}[2]{
	\thispagestyle{empty}
	\begin{center}
		\bfseries

		{\Large Московский авиационный институт\\ (национальный исследовательский университет)

		}

		\vspace{48pt}

		{\large
			Институт №8 ``Информационные технологии и прикладная математика''
			Кафедра 806 ``Вычислительная математика и программирование''
		}

		\vspace{36pt}

		{Лабораторная работа №#1  \\
			По курсу ``Программирование графических процессоров''
		}
		\vspace{48pt}

		{#2}

	\end{center}

	\vspace{72pt}

	\begin{flushright}
		\begin{tabular}{rl}
			Выполнил:      & П.\, А. Милько      \\
			Группа:        & М8О-408Б-17         \\
			Преподаватели: & К.Г. Крашенинников, \\
			               & А.Ю. Морозов.       \\
		\end{tabular}
	\end{flushright}

	\vfill

	\begin{center}
		\bfseries
		Москва\\
		\the\year
	\end{center}
	\newpage
	\setcounter{page}{1}
}

\newcommand{\nvidia}[0]{
	\se{Программное и аппаратное обеспечение}
	
	TODO
}
\renewcommand{\cource}{Параллельная обработка данных}

\begin{document}
\makemytitlepage{1}{Message Passing Interface (MPI)}

\se{Цель работы}

Знакомство с технологией MPI. Реализация метода Якоби.
Решение задачи Дирихле для уравнения Лапласа в трехмерной области с граничными
условиями первого рода.

\textbf{Вариант 8.} обмен граничными слоями через isend/irecv, контроль сходимости allreduce

\textbf{Входные данные.}

На первой строке заданы три числа: размер сетки
процессов. Гарантируется, что при запуске программы количество процессов будет
равно произведению этих трех чисел. На второй строке задается размер блока,
который будет обрабатываться одним процессом: три числа. Далее задается путь к
выходному файлу, в который необходимо записать конечный результат работы
программы и точность $\varepsilon$ . На последующих строках описывается задача: задаются
размеры области $l_x$, $l_y$ и $l_z$,
граничные условия: $u_down$, $u_up$, $u_left$, $u_right$, $u_front$ и $u_back$,
и начальное значение $u^0$.

\textbf{Выходные данные.}

В файл, определенный во входных данных, необходимо
напечатать построчно значения $(u_{1,1,1}, u_{2,1,1}, \cdots, u_{2,2,1}, \cdots, u_{n_x,n_y,n_z})$
в ячейках сетки в формате с плавающей запятой с семью знаками мантиссы.

\nvidia

\se{Метод решения}

\se{Исходный код}

% \listsource{../src}{main.cpp}
% \listsource{../src}{solver.hpp}
% \listsource{../src}{solver.cpp}
% \listsource{../src}{common.cpp}
% \listsource{../src}{exchange.hpp}
% \listsource{../src}{exchange.cpp}
% \listsource{../src}{problem.hpp}
% \listsource{../src}{problem.cpp}
% \listsource{../src}{grid/grid.hpp}
% \listsource{../src}{grid/grid.cpp}
% \listsource{../src}{dim3/dim3.hpp}
% \listsource{../src}{dim3/dim3.cpp}

\se{Результаты}

\se{Выводы}

Было интересно, но разрешать конфликты банков не очень весело.

\end{document}