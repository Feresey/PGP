\documentclass[12pt]{article}

\usepackage{amsmath}

\usepackage[utf8]{inputenc}
\usepackage[T1, T2A]{fontenc}
\usepackage{fullpage}
\usepackage{multicol, multirow}
\usepackage{tabularx}
\usepackage{ulem}
\usepackage{listings}
\usepackage[english, russian]{babel}
\usepackage{tikz}
\usepackage{pgfplots}
\usepackage{indentfirst}
\usepackage{noindentafter}
\usepackage{nonfloat}
\usepackage{ulem}
\usepackage{fancyhdr}
% \usepackage{courier}
% \usepackage{FiraMono}
\usepackage{color}
\usepackage{subcaption}
\usepackage{titlesec}


\parindent=1cm

\linespread{1}
\pgfplotsset{compat=1.16}
\newcommand{\se}[1]{\section*{\bf #1}}

\newcommand{\listsource}[2]{
	\subsection*{\textbf{#2}}
	\lstinputlisting[language=c++]{#1/#2}
}
\newcommand{\listtext}[2]{
	\subsection*{\textbf{#2}}
	\lstinputlisting[basicstyle=\tiny]{#1/#2}
}

\lstdefinestyle{empty}{language=c++,
	basicstyle=\scriptsize,
	showspaces=false,
	showstringspaces=false,
	showtabs=false, вариантов
	tabsize=4,
	breaklines=true,
	escapechar=@,
	numbers=none,
	frame=none,
	escapeinside={\%*}{*)},
	breakatwhitespace=false % переносить строки только если есть пробел
}

\lstdefinestyle{customc}{
	belowcaptionskip=1\baselineskip,
	breaklines=true,
	frame=L,
	xleftmargin=\parindent,
	numbers=left,
	showstringspaces=false,
	basicstyle=\footnotesize\ttfamily,
	keywordstyle=\bfseries\color{green!40!black},
	commentstyle=\itshape\color{purple!40!black},
	identifierstyle=\color{blue},
	stringstyle=\color{orange},
}

\lstset{escapechar=|,style=customc}

\pgfplotsset{compat=1.17}

\titleformat{\section}
{\normalfont\Large\bfseries}{\thesection.}{0.3em}{}

\titleformat{\subsection}
{\normalfont\large\bfseries}{\thesubsection.}{0.3em}{}

% \titlespacing{\section}{0pt}{*2}{*2}
% \titlespacing{\subsection}{0pt}{*1}{*1}
% \titlespacing{\subsubsection}{0pt}{*0}{*0}
% \lstloadlanguages{Lisp}
% \lstset{extendedchars=false,
% 	escapechar= |,
% 	breaklines=true,
% 	breakatwhitespace=true,
% 	keepspaces = true,
% 	tabsize=2
% }

\newcommand{\makemytitlepage}[2]{
	\thispagestyle{empty}
	\begin{center}
		\bfseries

		{\Large Московский авиационный институт\\ (национальный исследовательский университет)

		}

		\vspace{48pt}

		{\large
			Институт №8 ``Информационные технологии и прикладная математика''
			Кафедра 806 ``Вычислительная математика и программирование''
		}

		\vspace{36pt}

		{Лабораторная работа №#1  \\
			По курсу ``Программирование графических процессоров''
		}
		\vspace{48pt}

		{#2}

	\end{center}

	\vspace{72pt}

	\begin{flushright}
		\begin{tabular}{rl}
			Выполнил:      & П.\, А. Милько      \\
			Группа:        & М8О-408Б-17         \\
			Преподаватели: & К.Г. Крашенинников, \\
			               & А.Ю. Морозов.       \\
		\end{tabular}
	\end{flushright}

	\vfill

	\begin{center}
		\bfseries
		Москва\\
		\the\year
	\end{center}
	\newpage
	\setcounter{page}{1}
}

\newcommand{\nvidia}[0]{
	\se{Программное и аппаратное обеспечение}

	Редактор: NVIM v0.4.4

	GPU:

	\begin{tabular}{ll}
		Driver Version  & : 460.39          \\
		CUDA Version    & : 11.2            \\
		Product Name    & : GeForce GT 740M \\
		VBIOS Version   & : 80.28.22.00.23  \\
		Total Memory    & : 2004 MiB        \\
		Graphics Clocks & : 980 MHz         \\
		Memory Clocks   & : 920 MHz         \\
	\end{tabular}


}

\begin{document}
\makemytitlepage{4}{Работа с матрицами. Метод Гаусса.}

\se{Цель работы}

Использование объединения запросов к глобальной памяти.
Реализация метода Гаусса с выбором главного элемента по столбцу. Ознакомление с
библиотекой алгоритмов для параллельных расчетов Thrust.

\noindent
В качестве вещественного типа данных необходимо использовать тип данных
double. Библиотеку Thrust использовать только для поиска максимального элемента на
каждой итерации алгоритма. В вариантах(1,5,6,7), где необходимо сравнение по
модулю с нулем, в качестве нулевого значения использовать $10^{-7}$ . Все результаты
выводить с относительной точностью $10^{-10}$.

\textbf{Вариант 8.} Вариант на “два”. Быстрое транспонирование матрицы

\textbf{Входные данные.}

На первой строке заданы числа $n$ и $m$ -- размеры матрицы. В
следующих $n$ строках, записано по $m$ вещественных чисел -- элементы матрицы.
$n*m<=10^8$.

\textbf{Выходные данные.}

Необходимо вывести на $m$ строках, по $n$ чисел -- элементы
транспонированной матрицы.

\smallbreak

\subsubsection*{Дополнительные условия.}
\begin{itemize}
	\item Бесконфликтное использование разделяемой памяти.
	\item Объединение запросов к глобальной памяти.
\end{itemize}

\nvidia

\se{Метод решения}

В принципе в транспонировании нет ничего сложного.
Даже в официальной документации CUDA есть пример посвящённый работе с матрицами через разделяемую память.

Алгоритм основывается на знании работы варпа и особенностей работы с разделяемой и глобальной памятью.

\subsubsection*{Про варп}

Размер варпа фиксирован для всех устройств поддерживающих CUDA и равен 32 нитям выполнения.
Эти 32 нити выполняются одновременно, и соответственно доступ к памяти у них тоже одновременный.
Ну точнее одновременно к памяти обращается полуварп (16 нитей), а затем и второй полуварп.

\subsubsection*{Про разделяемую память}

У разделяемой памяти есть одна особенность --- она разделена на банки. Точнее на 32 банка.
Память распределяется по банкам 32-битными словами. Так первое слово попадёт в первый банк,
второе во второй банк, а 33 слово снова в первый банк.

И все работает прекрасно, пока каждый варп работает с отдельным банком.
Но иногда можно создать ситуацию, когда несколько варпов обращаются к разным элементам одного банка памяти.
Это называется конфликт банков памяти. Порядок конфликта --- количество нитей, которые залезли в один банк.

Чтобы осознать всю печаль ситуации достаточно узнать что в таком случае доступ к памяти происходит последовательно
для каждого варпа. То есть код выполняется в $n$ раз медленнее чем мог бы.

Есть правда исключение - если несколько нитей обращаются один и тот же элемент блока, то конфликта не будет.

\subsubsection*{Про транзакции}

Если читать память не последовательно то получится не очень хорошо.
Такое чтение может произойти если читать/писать матрицу по столбцам.
Тогда будут огромные прыжки по памяти и соответственно доступ будет медленным.

Но CUDA позволяет делать объединённые запросы к глобальной памяти --- транзакции.
Чтобы случилась транзакция все нити варпа должны обращаться к участку памяти последовательно.
Например, чтобы записать часть строки матрицы.

\subsection*{Собственно решение}

Размером варпа обусловлен размер разделяемой памяти и количество тредов в блоке.
Кеш будет хранить $32x32$ элемента.
В лучших традициях GPU сетка тредов в одном блоке будет тоже $32x32$.

\textbf{Шаг первый:} -- заполнение разделяемой памяти.
Тут всё просто -- прочитать из глобальной памяти и записать в разделяемую.
Если считывать по строкам, то получится даже эффективно.

\textbf{Шаг второй:} -- запись из разделяемой памяти в глобальную.
Тут уже становится интереснее, потому что чтобы запись шла транзакциями
результат надо записывать последовательно.
Этого легко добиться, если изменить индексацию разделяемой памяти.

Правда тогда уже разделяемая память будет считываться не последовательно, но это можно решить.

\subsection*{Разрешение конфликта}

Первый конфликт возникает если создать кеш втупую -- двумерный массив размером $32x32$ элемента.

Тогда банки памяти будут распределены таким образом:

\smallbreak

\begin{tabular}{crrrcc}
	0      & 1 & 2 & 3 & $\cdots$ & 31     \\
	0      & 1 & 2 & 3 & $\cdots$ & 31     \\
	0      & 1 & 2 & 3 & $\cdots$ & 31     \\
	0      & 1 & 2 & 3 & $\cdots$ & 31     \\
	\vdots &   &   &   & $\ddots$ & \vdots \\
	0      & 1 & 2 & 3 & $\cdots$ & 31     \\
\end{tabular}

При записи данных в разделяемую память варпы будут обращаться к разным банкам, записывая строки, и конфликта не будет.

А вот при чтении по столбцам возникнет конфликт 32 порядка, потому что все нити будут читать из одного банка.

Решением будет добавление фиктивного столбца --- он позволит столбцам матрицы попасть в разные банки памяти.

\smallbreak

\begin{tabular}{crrrcr|c}
	0        & 1 & 2 & 3 & $\cdots$ & 31 & 0        \\
	1        & 2 & 3 & 4 & $\cdots$ & 0  & 1        \\
	2        & 3 & 4 & 5 & $\cdots$ & 1  & 2        \\
	3        & 4 & 5 & 6 & $\cdots$ & 2  & 3        \\
	$\vdots$ &   &   &   & $\ddots$ &    & $\vdots$ \\
	31       & 0 & 1 & 2 & $\cdots$ & 30 & 31       \\
\end{tabular}

Да, так появится неиспользуемая память, но зато не будет конфликтов при работе как со строками, так и со столбцами.

\subsubsection*{Проблемы}

Я столкнулся с разным результатами профилирования на разных видеокартах.

На моей локальной $740M$ профилирование показывало 32 конфликта на один блок.

На удалённой $545$ с infway.ru конфликтов не обнаружилось.

Тут уж не знаю кто прав.
Наверное мой профилировщик, потому что я не учёл что тип \lstinline|double| имеет размер 64 бита,
а размер слова в разделяемой памяти 32 бита.

\smallbreak

Тогда я решил ещё раз изменить индексацию банков, но с расчётом на разделённый доступ к 16 элементам.
Если каждый дабл записался в 2 соседних банка.

\begin{tabular}[!htb]{cccccc|c|ccc}
	0        & 1 & 2 & 3 & $\cdots$ & 15       & 0        & $\cdots$ & 15       & 0        \\
	1        & 2 & 3 & 4 & $\cdots$ & 0        & 1        & $\cdots$ & 0        & 1        \\
	2        & 3 & 4 & 5 & $\cdots$ & 1        & 2        & $\cdots$ & 1        & 2        \\
	3        & 4 & 5 & 6 & $\cdots$ & 2        & 3        & $\cdots$ & 2        & 3        \\
	$\vdots$ &   &   &   &          & $\vdots$ & $\vdots$ &          & $\vdots$ & $\vdots$ \\
	15       & 0 & 1 & 2 & $\cdots$ & 14       & 15       & $\cdots$ & 14       & 15       \\
	\hline
	0        & 1 & 2 & 3 & $\cdots$ & 15       & 0        & $\cdots$ & 15       & 0        \\
	\hline
	0        & 1 & 2 & 3 & $\cdots$ & 15       & 0        & $\cdots$ & 15       & 0        \\
	1        & 2 & 3 & 4 & $\cdots$ & 0        & 1        & $\cdots$ & 0        & 1        \\
	2        & 3 & 4 & 5 & $\cdots$ & 1        & 2        & $\cdots$ & 1        & 2        \\
	3        & 4 & 5 & 6 & $\cdots$ & 2        & 3        & $\cdots$ & 2        & 3        \\
	$\vdots$ &   &   &   &          & $\vdots$ & $\vdots$ &          & $\vdots$ & $\vdots$ \\
	15       & 0 & 1 & 2 & $\cdots$ & 14       & 15       & $\cdots$ & 14       & 15       \\
\end{tabular}

Выделенные строка и столбец не используются, но вроде как выравнивают банки.
После такого трюка профилирование не показывает конфликта на обоих видеокартах.

По факту я применил тот же трюк, но из за того что рассчитывал на 16 элементов получилось 4 блока.

{
\listsource{../src}{transponse.cu}
}

\newpage

\se{Результаты}

\begin{table}[!htb]
\begin{minipage}{.49\linewidth}
\centering
\caption*{GPU}
\begin{tabular}{|l|r|}
	\hline
	size      & time       \\
	\hline

	2x2       & 000.007520 \\
	2x3       & 000.007552 \\
	2x1       & 000.012480 \\
	34x1      & 000.015008 \\
	1000x1000 & 002.258624 \\
	8192x8192 & 146.007523 \\
	\hline
\end{tabular}
\end {minipage} %
\begin{minipage}{.49\linewidth}
\centering
\caption*{CPU}
\begin{tabular}{|l|r|}
	\hline
	size      & time        \\
	\hline

	2x1       & 0.005551    \\
	2x2       & 0.007144    \\
	2x3       & 0.010589    \\
	34x1      & 0.013095    \\
	1000x1000 & 1.804195    \\
	8192x8192 & 6684.401827 \\
	\hline
\end{tabular}
\end {minipage}
\end{table}

\se{Профилирование}

\listtext{.}{log.nvprof}

\se{Выводы}

Было интересно, но разрешать конфликты банков не очень весело.

\end{document}