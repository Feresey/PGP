\documentclass[12pt]{article}

\usepackage{amsmath}

\usepackage[utf8]{inputenc}
\usepackage[T1, T2A]{fontenc}
\usepackage{fullpage}
\usepackage{multicol, multirow}
\usepackage{tabularx}
\usepackage{ulem}
\usepackage{listings}
\usepackage[english, russian]{babel}
\usepackage{tikz}
\usepackage{pgfplots}
\usepackage{indentfirst}
\usepackage{noindentafter}
\usepackage{nonfloat}
\usepackage{ulem}
\usepackage{fancyhdr}
% \usepackage{courier}
% \usepackage{FiraMono}
\usepackage{color}
\usepackage{subcaption}
\usepackage{titlesec}


\parindent=1cm

\linespread{1}
\pgfplotsset{compat=1.16}
\newcommand{\se}[1]{\section*{\bf #1}}

\newcommand{\listsource}[2]{
	\subsection*{\textbf{#2}}
	{\footnotesize
		\lstinputlisting[language=c++]{#1/#2}
	}
}

\lstdefinestyle{empty}{language=c++,
	basicstyle=\scriptsize,
	showspaces=false,
	showstringspaces=false,
	showtabs=false, вариантов
	tabsize=4,
	breaklines=true,
	escapechar=@,
	numbers=none,
	frame=none,
	escapeinside={\%*}{*)},
	breakatwhitespace=false % переносить строки только если есть пробел
}

\lstdefinestyle{customc}{
	belowcaptionskip=1\baselineskip,
	breaklines=true,
	frame=L,
	xleftmargin=\parindent,
	language=c++,
	numbers=left,
	showstringspaces=false,
	basicstyle=\footnotesize\ttfamily,
	keywordstyle=\bfseries\color{green!40!black},
	commentstyle=\itshape\color{purple!40!black},
	identifierstyle=\color{blue},
	stringstyle=\color{orange},
}

\lstset{escapechar=@,style=customc}

\pgfplotsset{compat=1.17}

\titleformat{\section}
{\normalfont\Large\bfseries}{\thesection.}{0.3em}{}

\titleformat{\subsection}
{\normalfont\large\bfseries}{\thesubsection.}{0.3em}{}

% \titlespacing{\section}{0pt}{*2}{*2}
% \titlespacing{\subsection}{0pt}{*1}{*1}
% \titlespacing{\subsubsection}{0pt}{*0}{*0}
% \lstloadlanguages{Lisp}
% \lstset{extendedchars=false,
% 	escapechar= |,
% 	breaklines=true,
% 	breakatwhitespace=true,
% 	keepspaces = true,
% 	tabsize=2
% }

\newcommand{\makemytitlepage}[2]{
	\thispagestyle{empty}
	\begin{center}
		\bfseries

		{\Large Московский авиационный институт\\ (национальный исследовательский университет)

		}

		\vspace{48pt}

		{\large
			Институт №8 ``Информационные технологии и прикладная математика''
			Кафедра 806 ``Вычислительная математика и программирование''
		}

		\vspace{36pt}

		{Лабораторная работа №#1  \\
			По курсу ``Программирование графических процессоров''
		}
		\vspace{48pt}

		{#2}

	\end{center}

	\vspace{72pt}

	\begin{flushright}
		\begin{tabular}{rl}
			Выполнил:      & П.\, А. Милько      \\
			Группа:        & М8О-408Б-17         \\
			Преподаватели: & К.Г. Крашенинников, \\
			               & А.Ю. Морозов.       \\
		\end{tabular}
	\end{flushright}

	\vfill

	\begin{center}
		\bfseries
		Москва\\
		\the\year
	\end{center}
	\newpage
	\setcounter{page}{1}
}

\newcommand{\nvidia}[0]{
	\se{Программное и аппаратное обеспечение}
	
	TODO
}

\begin{document}
\makemytitlepage{3}{Сортировка чисел на GPU. Свертка, сканирование, гистограмма.}

\se{Цель работы}

Ознакомление с фундаментальными алгоритмами GPU: свертка
(reduce), сканирование (blelloch scan) и гистограмма (histogram). Реализация одной из
сортировок на CUDA. Использование разделяемой и других видов памяти.
Исследование производительности программы с помощью утилиты nvprof
(обязательно отразить в отчете).

\textbf{Вариант 5.} Сортировка чет-нечет.

\textbf{Входные данные.}
В первых четырех байтах записывается целое число n --
длина массива чисел, далее следуют n чисел типа заданного вариантом.

\textbf{Выходные данные.}

В бинарном виде записывают n отсортированных по
возрастанию чисел.


Требуется реализовать блочную сортировку чет-нечет для чисел типа int.
Должны быть реализованы:

\begin{itemize}
	\item Алгоритм чет-нечет сортировки для предварительной сортировки
	      блоков.
	\item Алгоритм битонического слияния, с использованием разделяемой
	      памяти.
\end{itemize}

Ограничения: $n <= 16 * 10^6$

\nvidia

\se{Метод решения}

Очень долго не понимал как соединить 2 сортировки, пока не осознал битоническое слияние.

Массив делится на блоки фиксированного размера, на каждом блоке отрабатывает сортировка чёт-нечёт.
Этот трюк даёт $n$ неубывающих последовательностей.

Сначала я хотел сделать битоническое слияние для двух таких получившихся последовательностей,
но упёрся в то что на последней итерации надо будет запускать битонику для всего массива.

Тогда другой вариант показался более привлекательным - сливать половинки блоков.
То есть вторую половину от предыдущего блока и первую от последующего.
После этого будет $n-1$ отсортированных блоков, а так же половинка блока в начале массива и половинка в конце.
Вторая итерация так же берёт половинки блоков, но начинается с начала массива, а не с серидины первого блока.

Так, если предположить худший вариант - блок с самыми малыми значениями последний, а с самыми большими первый,
то за $n$ итераций первый блок дойдёт до своего места, как и последний.
Блок будет двигаться по массиву как гусеница, за каждую итерацию одна его половинка будет двигаться на своё место.
Если какие-то элементы оказались разбросаны по массиву, то они встанут на своё место при проходе блока по массиву.

{
\listsource{../src}{main.cu}
}

\newpage

\se{Результаты}

\begin{table}[!htb]
\begin{minipage}{.49\linewidth}
\centering
\caption*{CPU}
\begin{tabular}{|l|r|}
	\hline
	n       & time         \\
	\hline

	10      & 0.623824     \\
	8123    & 9.101413     \\
	10000   & 13.736742    \\
	100000  & 774.255989   \\
	100500  & 789.445012   \\
	1000000 & 70896.706258 \\
	\hline
\end{tabular}
\end {minipage} %
\begin{minipage}{.49\linewidth}
\centering
\caption*{GPU}
\begin{tabular}{|l|r|}
	\hline
	n       & time        \\
	\hline

	10      & 269.320362  \\
	8123    & 247.722819  \\
	10000   & 178.349589  \\
	100000  & 193.055952  \\
	100500  & 204.694692  \\
	1000000 & 1527.011376 \\
	\hline
\end{tabular}
\end {minipage}
\end{table}

\se{Профилирование}

\listtext{.}{log.nvprof}

Легко видеть что чет-нечет сортировка получилась не очень.
Хотя 92\% времени и ушло на битоническую сортировку,
но на малых массивах хватает одного запуска чет-нечетной, которую можно было ускорить как минимум на треть.

\se{Выводы}

По результатам профилирования можно заметить насколько сильно предварительная сортировка ускорила процесс.
Разница по времени выполнения между одной чёт-нечётной сортровкой и одной последующей битонической чуть больше чем
150'000 раз. Даже с учётом того что чет-нечетная написана отвратно результат впечатляет.

Тест с $n=16*10^6$ выполнялся на CPU около часа, но я не дождался. На GPU он отработал за 5 минут.

\end{document}